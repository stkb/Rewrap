% LaTeX sample text.
% Modified from sample document at http://overleaf.com

\documentclass[a4paper]{article}

%% Packages
\usepackage[english]{babel}
\usepackage[utf8x]{inputenc}
\usepackage[T1]{fontenc}

%% This is a long line but has no spaces, so shouldn't be wrapped
\usepackage[a4paper,top=3cm,bottom=2cm,left=3cm,right=3cm,marginparwidth=1.75cm]{geometry}


\title{Your Paper}
\author{You}

\begin{document}
\maketitle

\begin{abstract}
Your abstract.
\end{abstract}

\section{Introduction}

Your introduction goes here! Some examples of commonly used commands and features are listed below, to help you get started. If you have a question, please use the help menu (``?'') on the top bar to search for help or ask us a question. 

\section{Some examples to get started}

\begin{figure}
\centering
\includegraphics[width=0.3\textwidth]{frog.jpg}
\caption{\label{fig:frog}This frog was uploaded via the project menu.}
\end{figure}

\subsection{How to add Tables}

Use the table and tabular commands for basic tables --- see Table~\ref{tab:widgets}, for example.

\begin{table}
\centering
\begin{tabular}{l|r}
Item & Quantity \\\hline
Widgets & 42 \\
Gadgets & 13
\end{tabular}
\caption{\label{tab:widgets}An example table.}
\end{table}

\subsection{How to add Lists}

You can make lists with automatic numbering \dots

\begin{enumerate}
\item Like this,
\item and like this.
\end{enumerate}
\dots or bullet points \dots
\begin{itemize}
\item Like this,
\item and like this.
\end{itemize}

\subsection{How to add Citations and a References List}

You can upload a \verb|.bib| file containing your BibTeX entries, created with JabRef; or import your \href{https://www.overleaf.com/blog/184}{Mendeley}, CiteULike or Zotero library as a \verb|.bib| file. You can then cite entries from it, like this: \cite{greenwade93}. Just remember to specify a bibliography style, as well as the filename of the \verb|.bib|.